\documentclass[a4paper,12pt]{article}

\usepackage{amsmath,amssymb,multicol,tikz,enumitem}
\usepackage[margin=2cm]{geometry}
\usetikzlibrary{calc,shapes}

\pagestyle{empty}

\newcommand\N{\mathbf{N}}
\newcommand\Q{\mathbf{Q}}
\newcommand\R{\mathbf{R}}
\newcommand\Z{\mathbf{Z}}

\usepackage{array}
\newcolumntype{P}[1]{>{\centering\arraybackslash}p{#1}}
\newcommand\indd{${}$\hspace{20pt}}

\begin{document}

\begin{center}
\parbox{3.5cm}{\textbf{Data Structures}} \hfill {\bf\Huge Worksheet 8} \hfill \parbox{3.5cm}{\flushright\textbf{BITL2}} \\[5pt]
\rm\small 25 November 2021
\end{center}

\hrule\vspace{2pt}\hrule

\begin{enumerate}

\item \textbf{Warm up:} Answer the following questions.
\begin{enumerate}
\item Why is hashing important?
\item What is the difference between a map and a hashing function?
\item In what cases is a rolling hash function the same as a regular hash function? 
\end{enumerate}

\vfill
\item Draw what happens when the keys $5, 28, 19, 15, 20, 33, 12, 17, 10,$ are insterted into a hash table with hash function $h(k) = k\pmod 9$, with collisions resolved by chaining. 

%\item rolling hash

\vfill
\item This question is about \textbf{string matching} algorithms.
\begin{enumerate}
\item Recall the naive string matching algorithm as you saw it in Discrete Structures. Consider the two strings
\[
s = \texttt{ambracadambrazampbra},
\hspace{1cm}
t = \texttt{amp}.
\]
How many characters will be compared when $t$ is searched for in $s$?
\item How many of those comparisons for \texttt{a} are pointless, because you already know the character is not \texttt{a}?
\item To fix the problem in part (b), for every string \texttt{s} we define the \textbf{prefix} function $\pi_{\texttt{s}}\colon \Z_{\geqslant 0} \to \Z_{\geqslant 0}$, given by
\begin{align*}
\pi_{\texttt{s}}(k) & = \max_{\ell<k} \left\{\texttt{s}[:\ell] = \texttt{s}[k-\ell:k]\right\} \\
& = \max_\ell \left\{\texttt{s}[:k][:\ell] = \texttt{s}[:k][-\ell:]\right\}.
\end{align*}
Find the values of $\pi_{\texttt{s}}(k)$ for each $k=0,\dots,\textup{len}(\texttt{s})$ for the strings
\begin{multicols}{4}
\begin{enumerate}
\item \texttt{grebulon}
\item \texttt{aaaaaaaaba}
\item \texttt{abaaaaaaaa}
\item \texttt{catercatcat}
\end{enumerate}
\end{multicols}
\item Suppose you are given a sequence of nonnegative integers $a_1,\dots,1_\ell$. Describe what conditions the sequence must meet to correspond to the values $\pi_{\texttt{s}}(1),\dots,\pi_{\texttt{s}}(\ell)$ of a string \texttt{s} of length $\ell$. How would you construct \texttt{s}?
\end{enumerate}
\end{enumerate}

\vfill
\vfill

\end{document}