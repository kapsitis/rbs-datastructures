\documentclass[a4paper,12pt]{article}

\usepackage{amsmath,amssymb,multicol,tikz,enumitem}
\usepackage[margin=2cm]{geometry}
\usetikzlibrary{calc,shapes}

\pagestyle{empty}

\newcommand\Q{\mathbf{Q}}
\newcommand\R{\mathbf{R}}
\newcommand\Z{\mathbf{Z}}

\usepackage{array}
\newcolumntype{P}[1]{>{\centering\arraybackslash}p{#1}}

\begin{document}

\begin{center}
\parbox{3.5cm}{\flushleft\bf Data Structures\linebreak BITL2} \hfill {\bf\Huge Worksheet 1} \hfill \parbox{3.5cm}{\flushright\bf Fall 2021} \\[8pt]
\rm\small 2 September 2020
\end{center}

\hrule\vspace{2pt}\hrule

\begin{enumerate}

\item \textbf{Warm up:} Answer the following True / False questions.
\begin{enumerate}
\item One byte is four bits.
\item One integer \texttt{int} is four bytes.
\item The sign of an integer \texttt{int} is given by the bit \texttt{-} in front of the \texttt{int}.
\item Since one byte can have 256 different values, 10 bytes can have $2560$ different values.
\end{enumerate}
\end{enumerate}

\vfill
\noindent
The next two problems refer to the following \texttt{C++} code. When compiled, the code on the left is a program called \texttt{power}, and the code on the right is a program called \texttt{readstop}.
\begin{multicols}{2}
{\fontsize{9}{10}\selectfont
\begin{verbatim}
    #include <iostream>
    using namespace std;
    int main() {
        float base;
        int exp;
        cin >> base;
        cin >> exp;
        float result = base;
        for (int i = 1; i < exp; i++) {
            result = result*base;
        }
        cout << result << "\n";
        return 0;
    }

\end{verbatim}
}
{\fontsize{9}{10}\selectfont
\begin{verbatim}
    #include <iostream>
    using namespace std;
    int main() {
        char x;
        bool stop;
        stop = false;
        while (!stop) {
            x = cin.peek();
            if (x == 'x') {
                stop = true;
            }
            cin >> x;
            cout << x;
        }
        cout << endl;
        return 0;
    }
\end{verbatim}
}
\end{multicols}

\vfill
\begin{enumerate}
\setcounter{enumi}{1}
\item This question is about the program \texttt{power}.
\begin{enumerate}
\item Complete the table below for a given input to the program \texttt{power}.
\begin{center}
\renewcommand\arraystretch{1.5}
\begin{tabular}{|r|c|c|c|c|c|c|c|c|c|}
\hline input & \texttt{2\ 4} &  \texttt{4\ 4\ 4} & \texttt{-2\ 4} & \texttt{2\ -4} & \texttt{-2\ -4} & \texttt{2.9\ 4} & \texttt{2.9 4.9} & \texttt{2E10 4}\\
\hline output &&&&&&&& \\
\hline
\end{tabular}
\end{center}
\item Recall that \texttt{float} has a limited range. What is the largest number \texttt{X} for which the input \texttt{X 2} will output the square of \texttt{X}?
\end{enumerate}

\vfill
\item This question is about the program \texttt{readstop}.
\begin{enumerate}
\item What will be output if a file with contents \texttt{dexterous} will be used as input?
\item Modify the code so that the \texttt{while} loop exits at the occurence of two sequential characters \texttt{ax}, but not at each separately.
\item Modify the code so that the \texttt{while} loop exits at the second occurence of \texttt{x}.
\item Modify the line containing \texttt{if} so that the \texttt{while} loops exits either if \texttt{x} is encountered, or if the end-of-file character is encountered. Hint: use the boolean \texttt{cin.eof()}.
\end{enumerate}


\clearpage

\item This question is about \textit{flowcharts}.
\begin{enumerate}
\item Write a program that corresponds to the following flowchart and uses \texttt{switch}.
\[
\begin{tikzpicture}
\node[circle,draw] (1) at (0,0) {start};
\node[rectangle,draw] (2) at (3,0) {\texttt{cin >> x}};
\node[diamond,draw] (3) at (6,0) {\texttt{x == 1}};
\node[diamond,draw] (4b) at (6,-3) {\texttt{x == 2}};
\node[diamond,draw] (4d) at (6,-6) {\texttt{x > 3}};
\node[rectangle,draw] (4a) at (10,0) {\texttt{x = x*x}};
\node[rectangle,draw] (4c) at (10,-3) {\texttt{x = x*x*x}};
\node[rectangle,draw] (4e) at (10,-6) {\texttt{x = x+x}};
\node[rectangle,draw] (5) at (13,-9) {\texttt{x = 2*x}};
\node[rectangle,draw] (6) at (13,-11) {\texttt{cout << x}};
\node[circle,draw] (7) at (13,-13) {end};
\foreach \x\y in {1/2, 2/3, 5/6, 6/7}{
  \draw[-latex] (\x) -- (\y);
}
\draw[-latex] (4a) -| (5);
\draw[-latex] (4d) |- node[left,pos=.2] {false} (5);
\draw[-latex] (3) to node[left] {false} (4b);
\draw[-latex] (4b) to node[left] {false} (4d);
\draw[-latex] (3) to node[above] {true} (4a);
\draw[-latex] (4b) to node[above] {true} (4c);
\draw[-latex] (4d) to node[above] {true} (4e);
\draw[-latex] (4c)--(13,-3);
\draw[-latex] (4e)--(13,-6);
\end{tikzpicture}
\]
\item Write a program for the same flowchart, but using \texttt{if} and without \texttt{switch}.
\vfill
\item What will the program ouput if 3 is input?
\vfill
\item Is it ever possible to get an odd number output?
\vfill
\item Find two different numbers that give the same output.
\end{enumerate}

\vfill
\item Write a \texttt{C++} program called \texttt{dropunits} that takes as input an integer, and outputs the same integer, but without the units (that is, as a multiple of 10). For example, if the input \texttt{145} is given, then the program will print out \texttt{140}.

\vfill


\end{enumerate}

%\end{enumerate}


\end{document}