\documentclass[a4paper,12pt]{article}

\usepackage{amsmath,amssymb,multicol,tikz,enumitem}
\usepackage[margin=2cm]{geometry}
\usetikzlibrary{calc}

\pagestyle{empty}

\newcommand\Q{\mathbf{Q}}
\newcommand\R{\mathbf{R}}
\newcommand\Z{\mathbf{Z}}

\usepackage{array}
\newcolumntype{P}[1]{>{\centering\arraybackslash}p{#1}}

\begin{document}

\begin{center}
\parbox{3.5cm}{\flushleft\bf Data Structures\linebreak BITL2} \hfill {\bf\Huge Worksheet 1} \hfill \parbox{3.5cm}{\flushright\bf Fall 2021} \\[8pt]
\rm\small 2 September 2020
\end{center}

\hrule\vspace{2pt}\hrule

\begin{enumerate}

\item \textbf{Warm up:} Answer the following True / False questions.
\begin{enumerate}
\item One byte is four bits.
\item One integer \texttt{int} is four bytes.
\item The sign of an integer \texttt{int} is given by the bit \texttt{-} in front of the \texttt{int}.
\item Since one byte can have 256 different values, 10 bytes can have $2560$ different values.
\end{enumerate}

\item Consider the following \texttt{C++} code, compiled as a program \texttt{power} .
{\fontsize{9}{10}\selectfont
\begin{verbatim}
    #include <iostream>
    using namespace std;
    int main() 
    {
        float base;
        int exp;
        cin >> base;
        cin >> exp;
        float result = base;
        for (int i = 1; i < exp; i++)
        {
            result = result*base;
        }
        cout << result << "\n";
        return 0;
    }
\end{verbatim}
}
\begin{enumerate}
\item Complete the table below for a given input to the program \texttt{power}.
\begin{center}
\begin{tabular}{|r|c|c|c|c|c|c|c|c|c|}
\hline input & \texttt{2\ 4} & \texttt{-2\ 4} & \texttt{2\ -4} & \texttt{-2\ -4} & \texttt{2.9\ 4} & \texttt{2.9 4.9} & \texttt{2E10 4}\\
\hline output &&&&& \\
\hline
\end{tabular}
\end{center}

\item Recall that \texttt{float} has a limited range. What is the largest number \texttt{X} for which the input \texttt{X 2} will output the square of \texttt{X}?
\end{enumerate}

\item Write a \texttt{C++} program called \texttt{dropunits} that takes as input an integer, and outputs the same integer, but without the units (that is, as a multiple of 10). For example, if the input \texttt{145} is given, then the program will print out \texttt{140}.

\end{enumerate}

%\end{enumerate}


\end{document}