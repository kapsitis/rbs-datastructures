\documentclass[a4paper,12pt]{article}

\usepackage{amsmath,amssymb,multicol,tikz,enumitem}
\usepackage[margin=2cm]{geometry}
\usetikzlibrary{calc,shapes}

\pagestyle{empty}

\newcommand\Q{\mathbf{Q}}
\newcommand\R{\mathbf{R}}
\newcommand\Z{\mathbf{Z}}

\usepackage{array}
\newcolumntype{P}[1]{>{\centering\arraybackslash}p{#1}}
\newcommand\indd{${}$\hspace{20pt}}

\begin{document}

\begin{center}
\parbox{3.5cm}{\textbf{Data Structures}} \hfill {\bf\Huge Worksheet 3} \hfill \parbox{3.5cm}{\flushright\textbf{BITL2}} \\[5pt]
\rm\small 16 September 2021
\end{center}

\hrule\vspace{2pt}\hrule

\begin{enumerate}

\item \textbf{Warm up:} Answer the following True / False questions.
\begin{enumerate}
\item If your code initiates a \texttt{class} with a constructor, you must delete it with a destructor.
\item Without changing any other code, using either of the lines in the function declaration will not produce any errors upon compilation:
\begin{center}
\texttt{void func(int this, char that)}
\hspace{1cm}
\texttt{void func(int\& this, char that)}
\end{center}
\item If the body of \texttt{try \{..\}} contains a complation error, then the body of  \texttt{catch \{..\}} will be executed. 
\item 
\end{enumerate}

\end{enumerate}

\end{document}