\documentclass[a4paper,12pt]{article}

\usepackage{amsmath,amssymb,multicol,tikz,enumitem}
\usepackage[margin=2cm]{geometry}
\usetikzlibrary{calc,shapes}

\pagestyle{empty}

\newcommand\N{\mathbf{N}}
\newcommand\Q{\mathbf{Q}}
\newcommand\R{\mathbf{R}}
\newcommand\Z{\mathbf{Z}}

\usepackage{array}
\newcolumntype{P}[1]{>{\centering\arraybackslash}p{#1}}
\newcommand\indd{${}$\hspace{20pt}}

\begin{document}

\begin{center}
\parbox{3.5cm}{\textbf{Data Structures}} \hfill {\bf\Huge Worksheet 3} \hfill \parbox{3.5cm}{\flushright\textbf{BITL2}} \\[5pt]
\rm\small 16 September 2021
\end{center}

\hrule\vspace{2pt}\hrule

\begin{enumerate}

\item \textbf{Warm up:} Answer the following True / False questions.
\begin{enumerate}
\item If your code initiates a \texttt{class} with a constructor, you must delete it with a destructor.
\item Without changing any other code, using either of the lines in the function declaration will not produce any errors upon compilation:
\begin{center}
\texttt{void func(int this, char that)}
\hspace{1cm}
\texttt{void func(int\& this, char that)}
\end{center}
\item If the body of \texttt{try \{..\}} contains a complation error, then the body of  \texttt{catch \{..\}} will be executed. 
\item Every function $f(n)$ is both $\Omega(1)$ and $O(e^n)$.
\item If a function is $O(1)$, then it must be constant.
\end{enumerate}

\item The \textit{time complexity} of an algorithm is $O(f(n))$ if as $n\to\infty$, where $n$ is the size of the input, the algorithm takes at most $M|f(n)|$ time, for some $M\in \R$. You are given Algorithm 1, which is $O(n^a)$, and Algorithm 2, which is $O(n^b)$, for $a,b\in \N$.
\begin{enumerate}
\item Give another function $f(n) \neq n^a$, so that Algorithm 1 is $O(f(n))$.
\item What is the complexity of the algorithm that:
\begin{enumerate}
\item first executes Algorithm 1, then Algorithm 2? 
\item exeutes Algorithm 1 and Algorithm 2 in parallel?
\end{enumerate}
\item Suppose that Algorithm 3 is $O(2^{n+1})$ and Algorithm 4 is $O(2^{2n})$. Are either / both / none of these algorithms $O(2^n)$?
\end{enumerate}

\vfill

\item Consider the pseudocode on the left, which takes as input a set of numbers $X = \{x_1,\dots,x_n\}$.
\newcommand\ind{${}$\hspace{10pt}}
\newcommand\yfac{1.5}
\begin{center}
\begin{tabular}{r l}
1 & \textbf{for} $i=n,n-1,\dots,2$: \\
2 & \ind \textbf{for} $j=1,2,\dots,i-1$: \\
3 & \ind \ind $x = x_j$ \\
4 & \ind \ind \textbf{if} $x > x_{j+1}$: \\
5 & \ind \ind \ind $x_j = x_{j+1}$ \\
6 & \ind \ind \ind $x_{j+1} = x$
\end{tabular}
\hspace{2cm}
\begin{tikzpicture}[scale=.8,baseline=-2cm]
\foreach \y in {0,1,2,3,4}{
  \draw (0,-\yfac*\y)--(4,-\yfac*\y) (0,1-\yfac*\y)--(4,1-\yfac*\y);
  \foreach \x in {0,...,4}{
    \draw (\x,-\yfac*\y)--(\x,1-\yfac*\y);
  }
  \node[anchor=east] at (-.5,.5-\yfac*\y) {step $\y$:};
}
\foreach \x\n in {0/1, 1/3, 2/4, 3/2}{
  \node at (\x+.5,.5) {$\n$};
}
\end{tikzpicture}
\end{center}
\begin{enumerate}
\item How many times is line 3 called?
\item What is an upper bound on the number of times line 5 is called?
\item In the boxes on the right above, starting with $X$ as given in step 0, write what $X$ looks like every time the order of its elements changes.
\item What do you think the code does to $X$?
\end{enumerate}

\clearpage

\item This question is about the following uncompiled \texttt{C++} code:
{\fontsize{9}{10}\selectfont
\begin{verbatim}
 1       #include <iostream>
 2       #include <vector>
 3       #include <string>
 4       using namespace std;
 5
 6       class Player {
 7           \\ Your code here
 8       };
 9
10       bool player_compare(Player &A, Player &B) {
11           \\ Your code here
12       };
13
14       int main() {
15           vector<Player> gg{
16               {"Chad", 10}, {"Brad", 15}, {"Mad", 7}, {"Dad", 20}, {"Tad", 12}
17           };
18           Player& currentwinner = gg[0];
19           for (vector<Player>::iterator it = gg.begin(); it != gg.end(); ++it) {
20               if (player_compare(*it, currentwinner)) {
21                   currentwinner = *it;
22               }
23           }
24           cout << currentwinner.name << endl;
25       }
\end{verbatim}
}
\vspace{5pt}
\begin{enumerate}
\item Fill in the mising spots in lines 7 and 11 to make this code print out the name of the player with the largest score.\\
\item If line 18 is changed to ``\texttt{const Player\& currentwinner = gg[0];}'', then compilation will produce (at least) two errors. On which lines will these errors occur, and why?\\
\item Modify the code so that instead of just the winner being printed, it prints out by how much the winner won, for example, ``\texttt{Dad wins by 5 points}''.
\end{enumerate}

\end{enumerate}

\end{document}