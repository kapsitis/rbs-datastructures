%% Generated by Sphinx.
\def\sphinxdocclass{report}
\documentclass[letterpaper,10pt,english,openany,oneside]{sphinxmanual}
\ifdefined\pdfpxdimen
   \let\sphinxpxdimen\pdfpxdimen\else\newdimen\sphinxpxdimen
\fi \sphinxpxdimen=.75bp\relax

\PassOptionsToPackage{warn}{textcomp}
\usepackage[utf8]{inputenc}
\ifdefined\DeclareUnicodeCharacter
% support both utf8 and utf8x syntaxes
  \ifdefined\DeclareUnicodeCharacterAsOptional
    \def\sphinxDUC#1{\DeclareUnicodeCharacter{"#1}}
  \else
    \let\sphinxDUC\DeclareUnicodeCharacter
  \fi
  \sphinxDUC{00A0}{\nobreakspace}
  \sphinxDUC{2500}{\sphinxunichar{2500}}
  \sphinxDUC{2502}{\sphinxunichar{2502}}
  \sphinxDUC{2514}{\sphinxunichar{2514}}
  \sphinxDUC{251C}{\sphinxunichar{251C}}
  \sphinxDUC{2572}{\textbackslash}
\fi
\usepackage{cmap}
\usepackage[T1]{fontenc}
\usepackage{amsmath,amssymb,amstext}
\usepackage{babel}



\usepackage{times}
\expandafter\ifx\csname T@LGR\endcsname\relax
\else
% LGR was declared as font encoding
  \substitutefont{LGR}{\rmdefault}{cmr}
  \substitutefont{LGR}{\sfdefault}{cmss}
  \substitutefont{LGR}{\ttdefault}{cmtt}
\fi
\expandafter\ifx\csname T@X2\endcsname\relax
  \expandafter\ifx\csname T@T2A\endcsname\relax
  \else
  % T2A was declared as font encoding
    \substitutefont{T2A}{\rmdefault}{cmr}
    \substitutefont{T2A}{\sfdefault}{cmss}
    \substitutefont{T2A}{\ttdefault}{cmtt}
  \fi
\else
% X2 was declared as font encoding
  \substitutefont{X2}{\rmdefault}{cmr}
  \substitutefont{X2}{\sfdefault}{cmss}
  \substitutefont{X2}{\ttdefault}{cmtt}
\fi


\usepackage[Bjarne]{fncychap}
\usepackage{sphinx}

\fvset{fontsize=\small}
\usepackage{geometry}


% Include hyperref last.
\usepackage{hyperref}
% Fix anchor placement for figures with captions.
\usepackage{hypcap}% it must be loaded after hyperref.
% Set up styles of URL: it should be placed after hyperref.
\urlstyle{same}

\addto\captionsenglish{\renewcommand{\contentsname}{Contents:}}

\usepackage{sphinxmessages}
\setcounter{tocdepth}{1}

\usepackage{mystyle}

\title{Assignment08}
\date{Dec 15, 2021}
\release{}
\author{RBS}
\newcommand{\sphinxlogo}{\vbox{}}
\renewcommand{\releasename}{}

\begin{document}

\pagestyle{empty}

\pagestyle{plain}

\pagestyle{normal}
\phantomsection\label{\detokenize{index::doc}}



\chapter{Written Assignment 08}
\label{\detokenize{assignment08:written-assignment-08}}\label{\detokenize{assignment08::doc}}
\sphinxAtStartPar
Let \(G(V,E)\) be a directed graph. Let \(w:E\rightarrow{}\mathbf{Z}\)
be a function assigning integer weights to all the graph’s edges and let \(s \in V\) be
the source vertex.
Every vertex \(v \in V\) stores \(v.d\) \textendash{} the current estimate of
the distance from the source. A vertex also stores \(v.p\) \textendash{}
its “parent” (the last vertex on the shortest path before reaching \(v\)).
Bellman\sphinxhyphen{}Ford algorithm to find the minimum distance from \(s\) to all the other
vertices is given by the following pseudocode:

\begin{DUlineblock}{0em}
\item[] \(\text{\sc BellmanFord}(G,w,s)\):
\item[]
\begin{DUlineblock}{\DUlineblockindent}
\item[] \sphinxstylestrong{for} \sphinxstylestrong{each} vertex \(v \in V\):      \sphinxstyleemphasis{(initialize vertices to run shortest paths)}
\item[]
\begin{DUlineblock}{\DUlineblockindent}
\item[] \(v.d = \infty\)
\item[] \(v.p = \text{\sc Null}\)
\end{DUlineblock}
\item[] \(s.d = 0\)      \sphinxstyleemphasis{(the distance from source vertex to itself is 0)}
\item[] \sphinxstylestrong{for} \(i=1\) \sphinxstylestrong{to} \(|V|-1\)      \sphinxstyleemphasis{(repeat} \(|V|-1\) \sphinxstyleemphasis{times)}
\item[]
\begin{DUlineblock}{\DUlineblockindent}
\item[] \sphinxstylestrong{for} \sphinxstylestrong{each} edge \((u,v) \in E\)
\item[]
\begin{DUlineblock}{\DUlineblockindent}
\item[] \sphinxstylestrong{if} \(v.d > u.d + w(u,v)\):      \sphinxstyleemphasis{(relax an edge, if necessary)}
\item[]
\begin{DUlineblock}{\DUlineblockindent}
\item[] \(v.d = u.d + w(u,v)\)
\item[] \(v.p = u\)
\end{DUlineblock}
\end{DUlineblock}
\end{DUlineblock}
\end{DUlineblock}
\end{DUlineblock}

\begin{figure}[htbp]
\centering
\capstart

\noindent\sphinxincludegraphics[width=2.5in]{{bellman-ford-graph}.png}
\caption{A directed graph for Bellman\sphinxhyphen{}Ford Algorithm}\label{\detokenize{assignment08:id1}}\end{figure}

\sphinxAtStartPar
In this task the input graph is shown in Fig.1.
\begin{description}
\item[{\sphinxstylestrong{(A)}}] \leavevmode
\sphinxAtStartPar
In your graph use the vertex \(s=v_0\) as the \sphinxstyleemphasis{source vertex}
for Bellman\sphinxhyphen{}Ford algorithm.
Create a table showing the changes
to all the distances to the vertices of the given graph every time a successful edge
relaxing happens and some distance is reduced.
You should run \(n-1\) phases of the Bellman\sphinxhyphen{}Ford algorithm
(where \(n\) is the number of vertices). You can also stop earlier, if
no further edge relaxations can happen.

\begin{sphinxadmonition}{note}{Note:}
\sphinxAtStartPar
Please make sure to release the edges in the lexicographical order.
For example, in a single phase the edge \((v_1,v_4)\) is
relaxed before the edge \((v_2,v_1)\), since
\(v_1\) precedes \(v_2\).
\end{sphinxadmonition}

\item[{\sphinxstylestrong{(B)}}] \leavevmode
\sphinxAtStartPar
Summarize the result: For each vertex
tell what is its minimum distance from the source.
Also tell what is the shortest path how to get there.

\item[{\sphinxstylestrong{(C)}}] \leavevmode
\sphinxAtStartPar
Does the input graph contain negative cycles?
Justify your answer.

\end{description}

\sphinxAtStartPar
{\color{blue}
\sphinxstylestrong{Answer:}
\begin{description}
\item[{\sphinxstylestrong{(A)}}] \leavevmode
\sphinxAtStartPar
AA

\item[{\sphinxstylestrong{(B)}}] \leavevmode
\sphinxAtStartPar
BB

\item[{\sphinxstylestrong{(C)}}] \leavevmode
\sphinxAtStartPar
CC

\end{description}

}% end blue
\renewcommand{\indexname}{Index}
\printindex
\end{document}